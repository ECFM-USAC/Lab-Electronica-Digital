% Para iniciar una sección debe escribirse
%\section{Nombre de la sección}
% Lo anterior inmediatamente creará la sección y la numerará.

\section*{Objetivos}
\subsection*{Generales}
\begin{itemize}
    \item Introducir a los alumnos a la práctica de ensamblado de circuitos digitales
\end{itemize}

\subsection*{Específicos}
\begin{itemize}
    \item Demostrar la necesidad de mantener un estado estable de entrada (pull-up/pull-down)
    \item Acondicionar los circuitos para mantener un voltaje de operación óptimo
    \item Mostrar las operaciones lógicas básicas en funcionamiento en un circuito implementado físicamente
    \item Contrastar los diseños teóricos con los resultados experimentales de los circuitos implementados físicamente
\end{itemize}

\section{Introducción}
\subsection{Niveles lógicos}
Cuando se trabaja con lógica digital la representación de un estado se hace utilizando lógica binaria, la cuál requiere
de dos estados claramente definidos: $1$ y $0$.

A nivel físico, dichos estados pueden a su vez representarse con distintos tipos de magnitudes medibles, tales como 
un voltaje constante, una corriente, una temperatura, o un valor de luminosidad.

Asimismo, tal y como se ha explicado en clase, dependiendo de la tecnología utilizada (TTL, CMOS, LVCMOS, etc) los niveles
lógicos se representan con un nivel de voltaje que obedece a los siguientes umbrales:

\begin{table}[H]
    \centering
    \begin{tabular}{|c|c|c|}
        \hline
        \textbf{Estado} & \textbf{CMOS} & \textbf{TTL}  \\ \hline
        \textbf{0}      & 0.0 V - 1.5 V & 0.0 V - 0.8 V \\ \hline
        \textbf{1}      & 3.5 V - 5.0 V & 2.0 V - 5.0 V \\ \hline
    \end{tabular}
    \caption{Umbrales de voltaje para niveles lógicos}
    \label{table:umbralesLogicos}
\end{table}





\section{Desarrollo Experimental}
